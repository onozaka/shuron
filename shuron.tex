\documentclass[a4paper,10pt,oneside,openany,uplatex]{jsbook}

%図表の個数などの設定.
\setcounter{topnumber}{4}
\setcounter{bottomnumber}{4}
\setcounter{totalnumber}{4}
\setcounter{dbltopnumber}{3}

\renewcommand{\topfraction}{.95}
\renewcommand{\bottomfraction}{.90}
\renewcommand{\textfraction}{.05}
\renewcommand{\floatpagefraction}{.95}

%使用するパッケージを記述.
\usepackage{amsmath, amssymb} %複雑な数式などを打つときに使用.
\usepackage{bm} %数式環境内で太字を使うときに便利.
\usepackage{graphicx} %画像を挿入したり,テキストや図の拡大縮小・回転を行う.
\usepackage{subfigure} %図を並べる(今はsubfigとかsubcaptionとかが推奨らしい.よく知らない)
\usepackage{verbatim} %入力どおりの出力を行う.
\usepackage{ascmac} %テキストを枠で囲んだりできるが,微小なズレがでたりする.
\usepackage{makeidx} %索引を作成できる.
\usepackage{dcolumn} %表の数値を小数点で桁を揃える.
\usepackage{lscape} %図表を90度横に倒して配置する.
\usepackage{setspace} %行間調整.

%余白の設定.
\setlength{\textwidth}{150truemm}      % テキスト幅: 210-(30+30)=150mm
\setlength{\fullwidth}{\textwidth}     % ページ全体の幅
\setlength{\oddsidemargin}{30truemm}   % 左余白
\addtolength{\oddsidemargin}{-1truein} % 左位置デフォルトから-1inch
\setlength{\topmargin}{15truemm}       % 上余白
\setlength{\textheight}{242truemm}     % テキスト高さ: 297-(25+30)=242mm
\addtolength{\topmargin}{-1truein}     % 上位置デフォルトから-1inch


\renewcommand{\labelenumi}{(\arabic{enumi}) } %enumerate環境を1. 2.の形式から(1) (2)の形式へ変更(文書全体).

%\setcounter{tocdepth}{2} %項レベルまで目次に反映させるコマンド.

\begin{document}


\begin{titlepage}
\begin{spacing}{2.3}

\begin{center}
\vspace*{80truept}
{\huge 平成30年度(2018年度)修士論文}\\
\vspace{30truept}
{\huge タイトル}\\ % タイトル
{\LARGE ------サブタイトル------}\\ % サブタイトル(なければコメントアウト)
\vspace{100truept}
{\large 京都大学大学院\ 理学研究科\ 物理学第二教室\ 宇宙線研究室}\\ % 所属
{\LARGE 小野坂 健}\\ % 著者
{\LARGE 2019年1月1日\ 提出}\\ % 提出日

\end{center}

\end{spacing}
\end{titlepage} %「cover.tex」というファイルを同じディレクトリ上においておくこと.
%
\frontmatter

\begin{abstract} % 論文主旨
\begin{center}
{\large Abstract}\\
\end{center}
MeV領域のガンマ線を観測することで超新星残骸による元素合成などの解明が可能である。しかしこの領域はX線やGeV/TeVガンマ線の領域に比べて1桁感度が悪く、十分な観測が行われていない。そこで我々は従来の10倍の検出感度を目標とし、電子飛跡検出型コンプトンカメラ(Electron Tracking Compton Camera : ETCC)を開発している。この検出器は反跳電子の3次元飛跡とエネルギーを測定するガス検出器と、散乱ガンマ線の位置とエネルギーを測定するシンチレーションカメラから構成されている。これにより従来のコンプトンカメラでは測定できなかった反跳電子の飛跡情報を得ることによりコンプトン散乱を完全に再構成することができ、ガンマ線の到来方向を光子毎に一意に決定することが可能で高いバックグランド除去能力をもつ。我々は衛星搭載を目標としており、その前段階として気球実験計画SMILE(Sub Mev and MeV gamma-ray Imaging Loaded-on-Balloon Experiment)を立ち上げ、進行してきた。そして2018年4月にETCCの天体イメージング能力の実証試験として豪州にて気球実験を行なった。本論文ではこの気球実験で使用したETCCフライトモデルの課題であった消費電力と熱の問題を踏まえ、気球全体における電源システム、姿勢系センサーの性能評価、そして実際のフライトを経て得られたETCCの上空での姿勢情報の解析や、取得した荷電粒子のデータから東西効果を確認したことについて述べる。(使用バッテリー・持続時間。フライト時間・姿勢の様子、東西効果の優位度(?))

MeV天文学したい
でもうまくいってないよ
そこで我々はETCCを開発
ETCCはこういう特徴があってすごいよ
将来は衛星搭載して観測をしたいよ
でもその前に気球を使ってちゃんと観測をできるか気球を使って試験する計画を建てているよ
今回は天体イメージング能力の実証のためにオーストラリアで実験をしたよ
そのために前回のモデルから色々改良を加えたフライトモデルを作ったよ
上空ではこれこれこういう環境になるので、それでも動くようにバッテリー選びや熱設計をして環境試験を行ったよ
上空での姿勢を把握できるように姿勢系センサーを載せていて、それのテストなども行ったよ
実際の実験でもちゃんと動作してデータを取ることができたよ
姿勢の解析をしたよ
東西効果も見れたよ(?)
\end{abstract}

\tableofcontents %目次

 % 本文
 %本文は章や節ごとに分割して別々のファイルにして管理し,\input{}で挿入していくのがよいだろう.
\mainmatter

\chapter{はじめに}
\section{さらにはじめに}
\subsection{さらにさらにはじめに}

%あとがき
 \backmatter
\chapter{あとがき}
ありがとうございました.

%参考文献(bibtex不使用で,手動で入力する場合.字下げの形式は『社会学評論スタイルガイド』にしたがっている)
\def\bibindent{1.85em}
\begin{thebibliography}{99\kern\bibindent}
\makeatletter
\def\@biblabel#1{}
\let\old@bibitem\bibitem
\def\bibitem#1{\old@bibitem{#1}\leavevmode\kern-\bibindent}
\makeatother
\small
\bibitem{}
Granovetter, Mark. 1973. ``The Strength of Weak Ties." \textit{American Journal of Sociology} 78(6): 1360--80.
\end{thebibliography}
\end{document}